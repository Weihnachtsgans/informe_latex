\documentclass{article}
\usepackage[utf8]{inputenc}

\title{Documentación Turismo App}
\date{Abril 2018}

\usepackage{natbib}
\usepackage{graphicx}
\usepackage{imakeidx}
\usepackage{listings}
\usepackage{enumitem}

\renewcommand{\labelenumii}{\theenumii}
\renewcommand{\theenumii}{\theenumi.\arabic{enumii}.}
\makeindex

\begin{document}

\maketitle
\newpage
\tableofcontents
\newpage

\section{Java}
El archivo principal de la aplicación, es "turismo", a su vez, se divide en múltiples clases, donde cada una será explicada con más detalle a continuación.

\subsection{Clases}
    \subsubsection{Model}
        Es la clase principal. Todas las demás clases que tienen que interactuar con la base de datos, extienden ésta clase. Tiene dos functiones principales
        \begin{lstlisting}[language=Java]
public static ResultSet executeSelectQuery(String SQL,
    List<Object> parameters,
    Connection conn) throws SQLException{
        int index = 1;
        PreparedStatement stmt = conn.prepareStatement(SQL);
        stmt = setPreparedStatementValues(parameters, index, stmt);
        return stmt.executeQuery();
}
        \end{lstlisting}
        La cual ejecuta una operación de "select" en la base de datos. Recibe una lista de "Objetos" (tipo genérico), una conexión y la query como tal.
        Luego de ejecutar la query, retorna su "ResultSet".
        
        Ahora, la función que se encarga del resto de las operaciones de acceso a la base de datos es la siguiente.
        \begin{lstlisting}
public static Boolean executePostQuery(String SQL,
    List<Object> parameters,
    Connection conn) throws SQLException{
        int index = 1;
        PreparedStatement stmt = conn.prepareStatement(SQL);
        stmt = setPreparedStatementValues(parameters, index, stmt);
        return stmt.executeUpdate() > 0;
}
        \end{lstlisting}
        Recibe los mismos parámetros que la otra función, la diferencia es que en ésta se retorna un booleano, simbolizando que se hicieron cambios en la base de datos o no.
        
        Ambas funciones, llaman a ésta función.
        \begin{lstlisting}[language=Java]
private static PreparedStatement setPreparedStatementValues(
    List<Object> parameters,
    int index,
    PreparedStatement stmt) throws SQLException {
        for(Object ob: parameters){
            if(ob.getClass() == String.class){
                stmt.setString(index,(String)ob);
            }
            else if(ob.getClass() == int.class){
                stmt.setInt(index,(int)ob);
            }
            else if(ob.getClass() == Integer.class){
                stmt.setInt(index,(Integer)ob);
            }
            else if(ob.getClass() == Boolean.class){
                stmt.setBoolean(index,(Boolean)ob);
            }
            else if(ob.getClass() == float.class){
                stmt.setFloat(index,(float)ob);
            }
            else if(ob.getClass() == java.sql.Date.class){
                Date temp = (Date)ob;
                java.sql.Date t = new java.sql.Date(temp.getTime());
                stmt.setDate(index,t);
            }
            else if(ob.getClass() == java.util.Date.class){
                Date temp = (Date)ob;
                java.sql.Date t = new java.sql.Date(temp.getTime());
                stmt.setDate(index,t);
            }
            index++;
        }
        return stmt;
}
        \end{lstlisting}
        La cual se encarga re recibir todos los parámetros y construir un "Prepared Statement" válido con todos ellos. La razón de usar "Prepared Statements" y no usar un formateo de Strings solamente, es para evitar inyecciones SQL.
    \subsubsection{Similitudes entre todas las clases}
        Todas las clases cuentan con los siguientes métodos
        \begin{enumerate}
            \item Constructor
            \item Asignadores
            \item Find
            \item Save
        \end{enumerate}
        \paragraph{Constructores}\mbox{}\\
        Los constructores, como su nombre indica, se encargan de construir un nuevo objeto del tipo de la clase. La mayoría son públicos, a excepción de uno, que es privado y se llama únicamente cuando se una la función "Find".
        \paragraph{Asignadores}\mbox{}\\
        Se usan para asignar valores a los atributos de la clase, ya que todos son privados. Hay ciertos atributos (como por ejemplo, el de determinar si un usuario es administrador) que son privados, y sólo se pueden cambiar mediante contacto directo con la base de datos.
        \paragraph{Find}\mbox{}\\
        Es un método estático que retorna un objeto de la clase ya construido, a partir de los datos ya sumistrados. Si no se encuentra el elemento deseado en la bse de datos, se arroja una excepción que deberá ser manejada por el proceso principal.
        \paragraph{Save}\mbox{}\\
        Esta función se encarga de guardar (o actualizar) los datos actuales en la base de datos. Para determinar si hay que actualizar o insertar, se usa el valor que posee un atributo general en todas las clases: "isNew", el cual de determina si una clase fué creada usando el método "find" o manualmente.
        
    \subsubsection{Clases principales}
        Se manejan las siguientes clases para la aplicación.
        \begin{enumerate}
            \item Usuario
            \begin{enumerate}
                \item cliente
                \item proveedor
            \end{enumerate}
            \item transporte
            \begin{enumerate}
                \item ruta
            \end{enumerate}
            \item hospedaje
            \begin{enumerate}
                \item Campamento
                \item Hospedaje
                \item Posada
                \item Residencia
            \end{enumerate}
        \end{enumerate}
        \paragraph{cliente}\mbox{}\\
        Maneja todas las interacciones con la table "clientes" de la base de datos. Permite crear un cliente, así como buscar uno nuevo. Actualmente, al ingresar o registrarse, se hará como cliente
        \paragraph{proveedor}\mbox{}\\
        Permite interactuar con todos los elementos de la base de datos que tengan que ver con la tabla de proveedor. 
        \paragraph{Transporte}\mbox{}\\
        En esta sección, se maneja todo lo relacionado con las rutas. Actualmente, se permite la interacción total con la tabla de rutas de la base de datos.
        \paragraph{Hospedaje}\mbox{}\\
        En esta sección se manejan las interacciones con los distintos hospedajes disponibles, cada clase, hace alución al tipo de hospedaje con el que interviene.
        
        
\subsection{Uso}
Cada clase, puede manejarse de forma independiente. Cada una sólo depende de la clase "model", por lo tanto, pueden importarse y usarse según sean necesarias.
En la clase principal, Turismo, se manejan el login y el registro del usuario, así como las distintas intervenciones del usuario (ya sea cliente o administrador) con el sistema. 

\end{document}

